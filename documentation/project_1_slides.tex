\documentclass[
    %aspectratio=169,
]{beamer}

\usepackage[english]{babel}
\usepackage[utf8]{inputenc}
\usepackage[T1]{fontenc}
\usepackage{csquotes}
\usepackage{expl3,biblatex}

\addbibresource{bibliography.bib}

\usepackage{booktabs}
\usetheme{Madrid}
\usecolortheme{spruce}
\usepackage{svg}
\usepackage{graphicx,caption}
\usepackage{floatrow}

% set font
\usefonttheme{serif}

% to prevent the boxes from overlapping the logo at the lower right corner
\addtobeamertemplate{block begin}{%
  \setlength{\textwidth}{0.9\textwidth}%
}{}

\addtobeamertemplate{block alerted begin}{%
  \setlength{\textwidth}{0.9\textwidth}%
}{}

\addtobeamertemplate{block example begin}{%
  \setlength{\textwidth}{0.9\textwidth}%
}{}


% usage: \hl{text to emphasis as yellow background}
\usepackage{soul}
\makeatletter
\let\HL\hl
\renewcommand\hl{%
  \let\set@color\beamerorig@set@color
  \let\reset@color\beamerorig@reset@color
  \HL}
\makeatother


% usage: create outline pages before each section
\AtBeginSection[]
{
  \begin{frame}
    \frametitle{Table of Contents}
    \tableofcontents[currentsection]
  \end{frame}
}

% ---------------------------------------------------

% Title page information
\title[Mock Presentation]{Separating intended and unintended implicit learning effect in preschooler language development study}
\subtitle[Short Subtitle]{STATS 570/409 \\ Client: Prof. Carolyn Quam, Speech \& Hearing Sciences}
\author[Data Whisperers]{Lauren, Will, Betsy, and Ryan \texorpdfstring{\\}{, }}
\institute[PSU]{Portland State University}
\date{May 11, 2023}
\logo{\includesvg[height=1cm]{./uibeamer/logo/ui-stacked-gold-black.svg}}
\subject{Presentation Subject}
\keywords{the, presentation, keywords}

% --------------------------------------------------

\begin{document}

% make the title page
\begin{frame}%[plain]
\maketitle
\end{frame}

% make the outline page/table of contents
%\begin{frame}
%\frametitle{Table of Contents}
%\tableofcontents
%\end{frame}

% ----------------------------------------------------

\begin{frame}{Background}
\begin{itemize}
    \item Experimental study comparing implicit learning in preschoolers with DLD and TLD.
    \item Participants seemed to learn an alternation heuristic in addition to associating sound with food or drink.
\end{itemize}
\centering
\includegraphics[scale=0.245]{project 1 Final slides/images/experiment.png}
    %% #TODO add image of experimental scheme from conference poster
\end{frame}

% ------------------------------------------------------

\begin{frame}{Goals}
\begin{itemize}
    \item Determine how to separate effect of alternation heuristic from learning of auditory cue corresponding to food or drink.
    \item If effects could not be separated, document difficulty in separating effects and impact of confounding.
\end{itemize}

\end{frame}

% ----------------------------------------------------

\begin{frame}{Basic Data Description}
\begin{itemize}
    \item Experimental data in tabular form.
    \item Response variable is binary indicating correctness of participant's indication of monster's request for food or drink.
    \item Additional variables explored: response time,  trial number,  trial order,  sound type,  experimental phase,  alternation from previous trial,  and participant type (TLD and DLD).
    \item 26 participants of each type (TLD and DLD).
\end{itemize}
\end{frame}

% ------------------------------------------------------

\begin{frame}{Exploratory Data Analysis}
\begin{itemize}
  \item No obvious outliers were observed in response times, correctness percentage, or side-switching counts for either DLD or TLD participants.
  \item The number of participants in each type, trial order group combination was examined.
  \item The trial order group specifies the sequence of alternation of the stimuli in the training and tests.
\end{itemize}
%% #TODO: insert figure showing count
\end{frame}

% ------------------------------------------------------

\begin{frame}{Trial Order Count Plot}
\centering
\includegraphics[scale=0.265]{project 1 Final slides/images/trial_order_bar_plot.png}
\end{frame}

% ------------------------------------------------------

\begin{frame}{Response Time Plot: EDA}
\centering
\includegraphics[scale=0.245]{project 1 Final slides/images/Train_vs_Test_ResponseTime.png}
\end{frame}

% ------------------------------------------------------

\begin{frame}{Correctness Plot: EDA}
\centering
\includegraphics[scale=0.245]{project 1 Final slides/images/Train_vs_Test_CorrectnessPerc.png}
\end{frame}

% ------------------------------------------------------

\begin{frame}{Methods explored but determined unsuitable}
\begin{itemize}
    \item Treating individual trials as treatment periods in a crossover experimental design
	\begin{itemize}
		\item Would have involved employing a mixed effects model
		\item Determined unsuitable since carryover effects appear to be present between trials in testing phase of experiment
	\end{itemize}
    \item Generalized additive mixed models using the `mcgv` library
	\begin{itemize}
		\item Considered adding autoregressive terms to capture time varying treatment effect
		\item Determined time varying confounding effects (alternation sequence) biases time varying treatment effect estimates
	\end{itemize}
\end{itemize}
\end{frame}

% ------------------------------------------------------

\begin{frame}{Methods explored that can handle time varying confounding}
\begin{itemize}
    \item Causal inference methods (used in longitudinal studies)
	\begin{itemize}
		\item G-computation formula
		\item Inverse probability weighting estimation of marginal structural models
	 	\item G-estimation of structural nested models
	\end{itemize}
    \item Sequential conditional mean models with propensity scores
\end{itemize}
\end{frame}


% ------------------------------------------------------

\begin{frame}{Adapting Study Data to gfoRmula Package}
\begin{itemize}
    \item Specify the outcome type 
    \item Structure an input dataset to the main function gformula()
    \item Interpret output (gformula class object)
\end{itemize}
\centering
\includegraphics[scale=0.55]{project 1 Final slides/images/table.png}
\end{frame}

% ------------------------------------------------------


\begin{frame}{Conclusions}
Recommend experimenting with implementing:
\begin{itemize}
    \item G-computation formula via R package `gfoRmula`
    \item Sequential conditional mean models with propensity scores via R package `geepack`
\end{itemize}
\end{frame}

% ------------------------------------------------------

\begin{frame}{Appendix}
\end{frame}

% ------------------------------------------------------

\begin{frame}{Causal Inference}
\begin{itemize}
    \item James Robins' generalized methods 
    \item Originally developed for addressing problems with longitudinal studies in medicine and epidemiology 
    \item This project could extend these methods to a novel context 
\end{itemize}
\centering
\includegraphics[scale=0.55]{project 1 Final slides/images/logo.png}
\end{frame}

% ------------------------------------------------------

\end{document}
